\chapter*{Resumen}
El mundo de las redes neuronales está en auge. Poder simular el cerebro humano en un ordenador, parece ser uno de los hitos más prometedores de la informatica. Es cierto que este hito todavía no se ha conseguido, pero mediante algoritmos de machine learning, ya es posible entrenar máquinas para que aprendan de forma parecida a como lo haría nuestro cerebro. El objetivo de este Trabajo de Fin de Grado es poner en práctica estos algoritmos utilizando la librería Keras.\\\\
En primer lugar, se hará una breve introduccion al mundo de las redes neuronales. Se empezará por lo más básico, explicando que es una red neuronal y definiendo las partes mas importantes de su arquitectura. Una vez entendidos los conceptos básicos, se describirán los 3 tipos de redes neuronales mas extendidas actualmente debido a sus buenos resultados: Perceptrón multicapa, redes convolucionales, y redes LSTM.\\\\
En segundo lugar, se describirá Keras, una librería python de deep learning con la cual podremos diseñar nuestros propios modelos de redes neuronales. Se detallarán las clases y funciones más importantes, así como la gran cantidad de posibilidades que nos ofrece.\\\\
Por último, se aplicarán todos los conocimientos descritos anteriormente para diseñar cuatro tipos de redes neuronales que resolverán dos tipos de problemas distintos. En cuanto a problemas de clasificación de datos, veremos como clasificar distintas casas segun su precio a partir de cualidades que se han recogido previamente, y veremos como clasificacar imagenes en las que aparecen numeros, para determinar que numero es el que se representa en la imagen. Por otro lado, se llevarán a cabo dos problemas de predicción de datos. Mediante el primero, veremos como una maquina es capas de predecir si una persona tendrá diabetes a partir de sus datso médicos, y con el segundo, veremos como ésta es capaz de generar textos una vez haya sido entrenada con ellos.

\section*{Palabras Clave}
Redes neuronales, Keras, Tensorflow, perceptrón, redes convolucionales, redes LSTM, diabetes, housing, mnist, generador de texto.

\newpage
\chapter*{Abstract}
The world of neural networks is growing. Being able to simulate the human brain in a computer seems to be one of the most promising milestones in computing. It is true that it has not been achieved, but using machine learning algorithms, it is already possible to train machines to learn in a similar way as our brain. The objective of  this End-of-Grade Work is to implement these algorithms using the Keras library.\\
First of all, a brief introduction to the world of neural networks will be made. It will start with the basics, explaining what is a red neuron and defining the most important parts of its architecture. After understanding the basics, I will describe the three types of neural networks more rife due to their results: Multilayer Perceptron, Convolutional Networks, and LSTM Networks. \\\\
Secondly, I will describe Keras, a deep learning python library with which it is able to design our own neural network models. I will detail the classes and the most important functions, as well as the great amount of possibilities that this library offers us. \\\\
Finally, all previous knowledge is applied to design four types of neural networks that will solve two different types of problems. As for the problems of data classification, we will see how to classify several houses according to their price from qualities that have been previously collected, and we will see how to classify images of handwritten numbers, to determine which number is represented in the picture. On the other hand, data prediction problems will be addressed. Through the first example, we will be able to predict if a person will have diabetes or not analyzing their medical data, and with the second, we will see how a machine is able to generate texts once he has been trained with them.

\section*{Keywords}
Neural network, Keras, Tensorflow, perceptron, convolutional neural network, LSTM network, diabetes, housing, mnist, text generator.
