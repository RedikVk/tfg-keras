\chapter*{Resumen}
El mundo de las redes neuronales está en auge. Poder simular el cerebro humano en un ordenador, parece ser uno de los hitos más prometedores de la informatica. .........
En primer lugar, se hará una breve introduccion al mundo de las redes neuronales. Se empezará por lo mas básico, explicando que es una red neuronal y cuales son las partes mas importantes de su arquitectura, y se describirán los 3 tipos de redes neuronales mas extendidos actualmente debido a sus buenos resultados: Perceptrón multicapa, redes convolucionales, y redes LSTM.\\\\
En segundo lugar, se describirá Keras, una librería de deep learning con la cual podremos diseñar nuestros propios modelos de redes neuronales. Se detallarán las funciones más importantes, así como todas las posibilidades que nos ofrece.\\\\
Por último, se aplicarán todos los conocimientos descritos anteriormente para diseñar 4 tipos de redes neuronales que resolverán 4 problemas distintos.

\section*{Palabras Clave}
Redes neuronales, Keras, Tensorflow, Perceptrón, Redes convolucionales, Redes LSTM.