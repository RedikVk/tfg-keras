\chapter{Códigos fuente}
\label{Anexo:codigo}

\section{4.1. Problema de diabetes - Regresión logística}
\begin{lstlisting}[language=Python]
from keras.models import Sequential
from keras.layers import Dense
import numpy
# se elige una semilla aleatoria para reproducir el mismo ejemplo varias veces
numpy.random.seed(4)
# carga del dataset de diabetes en indios americanos
dataset = numpy.loadtxt("pima-indians-diabetes.csv", delimiter=",")
# se separa el dataset en X (entrada) e Y (salida)
X = dataset[:,0:8]
Y = dataset[:,8]
# creacion del modelo
model = Sequential()
model.add(Dense(12, input_dim=8, init='normal', activation='relu'))
model.add(Dense(8, init='normal', activation='relu'))
model.add(Dense(1, init='normal', activation='sigmoid'))
# compilacion del modelo
model.compile(loss='binary_crossentropy', optimizer='adam', metrics=['accuracy'])
# ajuste del modelo
model.fit(X, Y, nb_epoch=150, batch_size=10)
# evaluacion del modelo
eva = model.evaluate(X, Y)
print("%s: %.2f%%" % (model.metrics_names[1], eva[1]*100))

# calculo de predicciones
predicciones = model.predict(X)
# redondeo de predicciones
rounded = [round(x) for x in predicciones]
print(rounded)
\end{lstlisting}

\section{4.2. Problema Boston Housing - Regresión lineal }
\subsection{Modelo estándar}
\begin{lstlisting}[language=Python]
import numpy
import pandas
from keras.models import Sequential
from keras.layers import Dense
from keras.wrappers.scikit_learn import KerasRegressor
from sklearn.model_selection import cross_val_score
from sklearn.model_selection import KFold
from sklearn.preprocessing import StandardScaler
from sklearn.pipeline import Pipeline

# definicion del modelo
def baseline_model():
	# creacion del modelo
	model = Sequential()
	model.add(Dense(13, input_dim=13, init='normal', activation='relu'))
	model.add(Dense(1, init='normal'))
	# compilacion del modelo
	model.compile(loss='mean_squared_error', optimizer='adam')
	return model

# carga del dataset de boston-housing
dataframe = pandas.read_csv("housing.csv", delim_whitespace=True, header=None)
dataset = dataframe.values
# se separa el dataset en las variables X (entradas) e Y (salida)
X = dataset[:,0:13]
Y = dataset[:,13]
# se elige una semilla aleatoria para reproducir el mismo ejemplo varias veces
numpy.random.seed(4)
# evaluacion del modelo
estimator = KerasRegressor(build_fn=baseline_model, nb_epoch=100, batch_size=5, verbose=0)

kfold = KFold(n_splits=10, random_state=4)
results = cross_val_score(estimator, X, Y, cv=kfold)
print("Resultados: %.2f (%.2f) MSE" % (results.mean(), results.std()))
\end{lstlisting}

\subsection{Modelo con datos de entrada normalizados}
\begin{lstlisting}[language=Python]
import numpy
import pandas
from keras.models import Sequential
from keras.layers import Dense
from keras.wrappers.scikit_learn import KerasRegressor
from sklearn.model_selection import cross_val_score
from sklearn.model_selection import KFold
from sklearn.preprocessing import StandardScaler
from sklearn.pipeline import Pipeline

# definicion del modelo de red neuronal
def baseline_model():
	# creación del modelo
	model = Sequential()
	model.add(Dense(13, input_dim=13, init='normal', activation='relu'))
	model.add(Dense(1, init='normal'))
	# compilacion del modelo
	model.compile(loss='mean_squared_error', optimizer='adam')
	return model

# carga del dataset
dataframe = pandas.read_csv("housing.csv", delim_whitespace=True, header=None)
dataset = dataframe.values
# se separa el dataset en las variables X (entradas) e Y (salida)
X = dataset[:,0:13]
Y = dataset[:,13]
# se elige una semilla aleatoria para reproducir el mismo ejemplo varias veces
numpy.random.seed(4)
# se evalúa el modelo con el dataset normalizado
estimators = []
estimators.append(('standardize', StandardScaler()))
estimators.append(('mlp', KerasRegressor(build_fn=baseline_model, nb_epoch=50, batch_size=5, verbose=0)))
pipeline = Pipeline(estimators)

kfold = KFold(n_splits=10, random_state=4)
results = cross_val_score(pipeline, X, Y, cv=kfold)
print("Estandarizado: %.2f (%.2f) MSE" % (results.mean(), results.std()))
\end{lstlisting}

\subsection{Modelo más profundo}
\begin{lstlisting}[language=Python]
import numpy
import pandas
from keras.models import Sequential
from keras.layers import Dense
from keras.wrappers.scikit_learn import KerasRegressor
from sklearn.model_selection import cross_val_score
from sklearn.model_selection import KFold
from sklearn.preprocessing import StandardScaler
from sklearn.pipeline import Pipeline

# Definicion del modelo de red neuronal mas profundo
def larger_model():
	# Creacion del modelo
	model = Sequential()
	model.add(Dense(13, input_dim=13, init='normal', activation='relu'))
	model.add(Dense(6, init='normal', activation='relu'))
	model.add(Dense(1, init='normal'))
	# Compilacion del modelo
	model.compile(loss='mean_squared_error', optimizer='adam')
	return model

# carga del dataset
dataframe = pandas.read_csv("housing.csv", delim_whitespace=True, header=None)
dataset = dataframe.values
# se separa el dataset en las variables X (entradas) e Y (salida)
X = dataset[:,0:13]
Y = dataset[:,13]
# se elige una semilla aleatoria para reproducir el mismo ejemplo varias veces
numpy.random.seed(4)

estimators = []
estimators.append(('standardize', StandardScaler()))
estimators.append(('mlp', KerasRegressor(build_fn=larger_model, nb_epoch=50, batch_size=5, verbose=0)))
pipeline = Pipeline(estimators)

kfold = KFold(n_splits=10, random_state=4)
results = cross_val_score(pipeline, X, Y, cv=kfold)
print("Resultados: %.2f (%.2f) MSE" % (results.mean(), results.std()))
\end{lstlisting}

\subsection{Modelo más ancho}
\begin{lstlisting}[language=Python]
import numpy
import pandas
from keras.models import Sequential
from keras.layers import Dense
from keras.wrappers.scikit_learn import KerasRegressor
from sklearn.model_selection import cross_val_score
from sklearn.model_selection import KFold
from sklearn.preprocessing import StandardScaler
from sklearn.pipeline import Pipeline

# Definicion del modelo de red neuronal mas ancho
def wider_model():
	# Creacion del modelo
	model = Sequential()
	model.add(Dense(20, input_dim=13, init='normal', activation='relu'))
	model.add(Dense(1, init='normal'))
	# Compilacion del modelo
	model.compile(loss='mean_squared_error', optimizer='adam')
	return model

# carga del dataset
dataframe = pandas.read_csv("housing.csv", delim_whitespace=True, header=None)
dataset = dataframe.values
# se separa el dataset en las variables X (entradas) e Y (salida)
X = dataset[:,0:13]
Y = dataset[:,13]
# se elige una semilla aleatoria para reproducir el mismo ejemplo varias veces
numpy.random.seed(4)

estimators = []
estimators.append(('standardize', StandardScaler()))
estimators.append(('mlp', KerasRegressor(build_fn=wider_model, nb_epoch=100, batch_size=5, verbose=0)))
pipeline = Pipeline(estimators)

kfold = KFold(n_splits=10, random_state=4)
results = cross_val_score(pipeline, X, Y, cv=kfold)
print("Resultados: %.2f (%.2f) MSE" % (results.mean(), results.std()))
\end{lstlisting}

\section{4.3 MNIST - Perceptrón multicapa vs Red convolucional}
\subsection{Perceptrón multicapa}
\begin{lstlisting}[language=Python]
import numpy
from keras.datasets import mnist
from keras.models import Sequential
from keras.layers import Dense
from keras.layers import Dropout
from keras.utils import np_utils

# Definicion del modelo de red neuronal
def baseline_model():
	# Creacion del modelo
	model = Sequential()
	model.add(Dense(num_pixels, input_dim=num_pixels, init='normal', activation='relu'))
	model.add(Dense(num_classes, init='normal', activation='softmax'))
	# Compilacion del modelo
	model.compile(loss='categorical_crossentropy', optimizer='adam', metrics=['accuracy'])
	return model

	# se elige una semilla aleatoria para reproducir el mismo ejemplo varias veces
numpy.random.seed(4)

# carga del dataset
(X_train, y_train), (X_test, y_test) = mnist.load_data()

# -------------------------------------------------------flatten 28*28 images to a 784 vector for each image
num_pixels = X_train.shape[1] * X_train.shape[2]
X_train = X_train.reshape(X_train.shape[0], num_pixels).astype('float32')
X_test = X_test.reshape(X_test.shape[0], num_pixels).astype('float32')

# normalización de los valores de entrada de 0-255 a 0-1
X_train = X_train / 255
X_test = X_test / 255

# -------------------------------------------------------one hot encode outputs
y_train = np_utils.to_categorical(y_train)
y_test = np_utils.to_categorical(y_test)
num_classes = y_test.shape[1]

model = baseline_model()
# Ajuste del modelo
model.fit(X_train, y_train, validation_data=(X_test, y_test), nb_epoch=10, batch_size=200, verbose=2)
# Evaluacion del modelo
scores = model.evaluate(X_test, y_test, verbose=0)
print("Baseline Error: %.2f%%" % (100-scores[1]*100))
\end{lstlisting}

\subsection{Red convolucional}
\begin{lstlisting}[language=Python]
import numpy
from keras.datasets import mnist
from keras.models import Sequential
from keras.layers import Dense
from keras.layers import Dropout
from keras.layers import Flatten
from keras.layers.convolutional import Convolution2D
from keras.layers.convolutional import MaxPooling2D
from keras.utils import np_utils
from keras import backend as K
K.set_image_dim_ordering('th')

def baseline_model():
	# Creacion del modelo
	model = Sequential()
	model.add(Convolution2D(32, 5, 5, border_mode='valid', input_shape=(1, 28, 28), activation='relu'))
	model.add(MaxPooling2D(pool_size=(2, 2)))
	model.add(Dropout(0.2))
	model.add(Flatten())
	model.add(Dense(128, activation='relu'))
	model.add(Dense(num_classes, activation='softmax'))
	# Compilacion del modelo
	model.compile(loss='categorical_crossentropy', optimizer='adam', metrics=['accuracy'])
	return model

# se elige una semilla aleatoria para reproducir el mismo ejemplo varias veces
numpy.random.seed(4)

# Carga del dataset
(X_train, y_train), (X_test, y_test) = mnist.load_data()
# reshape to be [samples][pixels][width][height]
X_train = X_train.reshape(X_train.shape[0], 1, 28, 28).astype('float32')
X_test = X_test.reshape(X_test.shape[0], 1, 28, 28).astype('float32')

# normalización de los valores de entrada de 0-255 a 0-1
X_train = X_train / 255
X_test = X_test / 255
# -------------------------------------------------------one hot encode outputs
y_train = np_utils.to_categorical(y_train)
y_test = np_utils.to_categorical(y_test)
num_classes = y_test.shape[1]

model = baseline_model()
# Ajuste del modelo
model.fit(X_train, y_train, validation_data=(X_test, y_test), nb_epoch=10, batch_size=200, verbose=2)
# Evaluacion del modelo
scores = model.evaluate(X_test, y_test, verbose=0)
print("Baseline Error: %.2f%%" % (100-scores[1]*100))
\end{lstlisting}

\section{4.4 Generación de textos - Red LSTM}
\subsection{Entrenamiento de la red}
\begin{lstlisting}[language=Python]
import numpy
from keras.models import Sequential
from keras.layers import Dense
from keras.layers import Dropout
from keras.layers import LSTM
from keras.callbacks import ModelCheckpoint
from keras.utils import np_utils
# Se carga el texto y se convierten todos los caracteres a minúscula
filename = "kafka.txt"
raw_text = open(filename).read()
raw_text = raw_text.lower()
# Creacion del mapeo caracter-entero
chars = sorted(list(set(raw_text)))
char_to_int = dict((c, i) for i, c in enumerate(chars))
# Resumen del dataset de entrada
n_chars = len(raw_text)
n_vocab = len(chars)
print "Total Characters: ", n_chars
print "Total Vocab: ", n_vocab
# -------------------------------------------------------prepare the dataset of input to output pairs encoded as integers
seq_length = 100
dataX = []
dataY = []
for i in range(0, n_chars - seq_length, 1):
	seq_in = raw_text[i:i + seq_length]
	seq_out = raw_text[i + seq_length]
	dataX.append([char_to_int[char] for char in seq_in])
	dataY.append(char_to_int[seq_out])
n_patterns = len(dataX)
print "Total Patterns: ", n_patterns
# -------------------------------------------------------reshape X to be [samples, time steps, features]
X = numpy.reshape(dataX, (n_patterns, seq_length, 1))
# Normalización de los datos de entrada
X = X / float(n_vocab)
# -------------------------------------------------------one hot encode the output variable
y = np_utils.to_categorical(dataY)
# Definicion del modelo de red neuronal LSTM
model = Sequential()
model.add(LSTM(256, input_shape=(X.shape[1], X.shape[2]), return_sequences=True))
model.add(Dropout(0.2))
model.add(LSTM(256))
model.add(Dropout(0.2))
model.add(Dense(y.shape[1], activation='softmax'))
model.compile(loss='categorical_crossentropy', optimizer='adam')
# Definicion de los checkpoint (punto de guardado de pesos)
filepath="weights-improvement-{epoch:02d}-{loss:.4f}-bigger.hdf5"
checkpoint = ModelCheckpoint(filepath, monitor='loss', verbose=1, save_best_only=True, mode='min')
callbacks_list = [checkpoint]
# Ajuste del modelo
model.fit(X, y, nb_epoch=50, batch_size=64, callbacks=callbacks_list)
\end{lstlisting}

\subsection{Generacion de texto}
\begin{lstlisting}[language=Python]
import sys
import numpy
from keras.models import Sequential
from keras.layers import Dense
from keras.layers import Dropout
from keras.layers import LSTM
from keras.callbacks import ModelCheckpoint
from keras.utils import np_utils
# Se carga el texto y se convierten todos los caracteres a minúscula
filename = "kafka.txt"
raw_text = open(filename).read()
raw_text = raw_text.lower()
# Creacion del mapeo caracter-entero, y su inverso
chars = sorted(list(set(raw_text)))
char_to_int = dict((c, i) for i, c in enumerate(chars))
int_to_char = dict((i, c) for i, c in enumerate(chars))
# Resumen del dataset de entrada
n_chars = len(raw_text)
n_vocab = len(chars)
print "Total Characters: ", n_chars
print "Total Vocab: ", n_vocab
# -------------------------------------------------------prepare the dataset of input to output pairs encoded as integers
seq_length = 100
dataX = []
dataY = []
for i in range(0, n_chars - seq_length, 1):
	seq_in = raw_text[i:i + seq_length]
	seq_out = raw_text[i + seq_length]
	dataX.append([char_to_int[char] for char in seq_in])
	dataY.append(char_to_int[seq_out])
n_patterns = len(dataX)
print "Total Patterns: ", n_patterns
# reshape X to be [samples, time steps, features]
X = numpy.reshape(dataX, (n_patterns, seq_length, 1))
# Normalización de los datos de entrada
X = X / float(n_vocab)
# -------------------------------------------------------one hot encode the output variable
y = np_utils.to_categorical(dataY)
# Definicion del modelo de red neuronal LSTM
model = Sequential()
model.add(LSTM(256, input_shape=(X.shape[1], X.shape[2]), return_sequences=True))
model.add(Dropout(0.2))
model.add(LSTM(256))
model.add(Dropout(0.2))
model.add(Dense(y.shape[1], activation='softmax'))
# Carga de los pesos de la red a partir de los checkpoints generados previamente
filename = "weights-improvement-47-1.2219-bigger.hdf5"
model.load_weights(filename)
model.compile(loss='categorical_crossentropy', optimizer='adam')
# se elige una semilla aleatoria para reproducir el mismo ejemplo varias veces
start = numpy.random.randint(0, len(dataX)-1)
pattern = dataX[start]
print "Seed:"
print "\"", ''.join([int_to_char[value] for value in pattern]), "\""
# Generacion de caracteres
for i in range(1000):
	x = numpy.reshape(pattern, (1, len(pattern), 1))
	x = x / float(n_vocab)
	prediction = model.predict(x, verbose=0)
	index = numpy.argmax(prediction)
	result = int_to_char[index]
	seq_in = [int_to_char[value] for value in pattern]
	sys.stdout.write(result)
	pattern.append(index)
	pattern = pattern[1:len(pattern)]
\end{lstlisting}

\newpage \thispagestyle{empty} % Página vacía
