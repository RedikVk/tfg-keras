\chapter{Introducción}
\label{chap:intro}

\vspace{-0.2cm}

TODO: Referenciar imagenes\\

\lsection{¿Qué es deep learning?}
Deep Learning (aprendizaje profundo) es una rama de machine learning (aprendizaje automático) formada por un conjunto de algoritmos que intentan modelar abstracciones de alto nivel en datos, usando grafos profundos con múltiples capas de procesamiento. Estas capas de procesamiento pueden estar compuestas por transformaciones tanto lineales como no lineales.\\
TODO: Origen en modelos de RN de los años 80 y 90

\lsection{¿Qué podemos hacer mediante deep learning?}
TODO: De entrada, resolver problemas de clasificación y predicción...\\
Otros casos mas concretos podrían ser:
\textbf{1. Colorear imágenes en blanco y negro.}\\
Tradicionalmente, esta tarea ha sido llevada a cabo por el ser humano de manera manual, hasta que mediante el uso de deep learning, se han utilizado los objetos y contextos de las propias imágenes para ser coloreadas. Para ello, se necesitan grandes redes neuronales convolucionales con capas supervisadas, que recrean la imagen coloreada de la misma manera que lo haría un ser humano.\\
\\\textbf{2. Añadir sonido a películas mudas.}\\
Pongamos el ejemplo de querer recrear el sonido que hace un palo al golpear con una superficie. Si entrenamos el sistema utilizando una gran cantidad de vídeos donde se muestra el sonido que hace un palo al ser golpeado contra diferentes superficies, nuestra red neuronal asociará los frames del vídeo mudo con la información ya aprendida, y seleccionará el sonido que mejor se adapte a la escena.\\
\\\textbf{3. Traducciones automáticas.}\\
Pese a que la traducción de palabras, frases o textos lleva siendo posible desde hace muchos años, mediante la utilización de redes neuronales se han alcanzado resultados mucho mejores sobre todo en dos áreas: traducción de texto, y traducción de imágenes. En el caso de los textos por ejemplo, se ha pasado de traducir palabras sueltas, a analizar y entender la gramática y la conexión entre palabras, haciendo deducciones mucho mas precisas del significado global de la frase. Para ello, se usan redes LSTM. Para el caso de traducción de textos en imágenes se utilizan redes convoluciones, ya que son capaces de identificar letras, con ellas formar palabras, y a su vez formar un texto. En muchos contextos a esto se le llama traducción visual.\\
\\\textbf{4. Clasificación de objetos en fotografías.}\\
Esta funcionalidad consiste en detectar y clasificar uno o mas objetos de la escena de una fotografía. Eso se consigue entrenando grandes redes convolucionales mediante imágenes aisladas de objetos conocidos.\\
A partir de aquí, el siguiente punto es crear una palabra o frase que describa el contenido de la imagen. En 2014, hubo un "boom" de algoritmos que alcanzaron resultados impresionantes a la hora de resolver este problema. La mayoría de ellos usaban redes LSTM para convertir las etiquetas que se generan al detectar objetos en una imagen, en frases con sentido.\\
\\\textbf{5. Generación de textos.}\\
Esta tarea consiste en, dado una recopilación de textos, generar nuevos textos a partir de una palabra o una frase.\\
Una aplicación podría ser la generación de textos escritos a mano. La escritura a mano consiste en una serie de movimientos coordinados de un bolígrafo, en los que se crea texto. Mediante machine learning, podemos aprender la relación entre los movimientos del bolígrafo y las letras escritas, para generar nuevos ejemplos. Esta funcionalidad puede ser utilizada por médicos forenses, o especialistas en análisis de manuscritos, ya que es posible aprender una cantidad impresionante de estilos de escritura.\\
Otra posible aplicación sería la de generar nuevos textos con nuevas historias. Mediantes redes LSTM se ha conseguido aprender la relación entre los distintos elementos de un texto (letra, palabra, frase...), para después generar nuevos textos letra a letra o palabra a palabra. Los modelos son capaces de aprender como deletrear, puntuar, formar oraciones, e incluso copiar los estilos de escritura para generar dichos textos.\\
\\\textbf{6. Inteligencia artificial en videojuegos.}\\
Todos sabemos que la inteligencia artificial en los vídeos es una cosa que lleva existiendo desde hace mucho tiempo... En un shooter, la IA de la videoconsola sabe identificar donde está tu jugador, y con esa información, sabe a quien y donde disparar. Pero... ¿y si fuese posible analizar todos los píxeles de la pantalla? Mediante deep learning esto es relativamente sencillo, permitiendo a la IA rival tener mucha mas información y tomando así mejores decisiones. Un ejemplo nos lo encontramos en AlphaGo, una aplicación desarrollada por Google que ganó al campeón mundial de Go.\\\\
Otros ejemplos de aplicaciones de redes neuronales para resolver problemas actuales podrían ser: reconocimiento y traducción de discursos y charlas en tiempo real, foco automático en objetos en movimiento en fotografías, conversión automática de objetos en fotografías, respuestas automáticas a preguntas sobre objetos en fotografías... Como se puede observar, casi todas estas tareas se tratan de automatizar. Son tareas que el ser humano puede hacer manualmente, pero una maquina es capaz de aprender a hacerlo en mucho menos tiempo y con mucha mas eficiencia.




\lsection{Antecedentes y estado actual}

%\lsection{Motivación del proyecto}
%Ejemplo de referencia a la bibliografía~\cite{article:Ejemplo}.

%Ejemplo de imagen:
%\begin{figure}[h]
% \centerline{
%    \mbox{\includegraphics[width=3.00in]{images/logo_eps.eps}}
%  }
%  \caption{Ejemplo pie de figura 1}
%  \label{fig:norm_Daugman}
%\end{figure}

%\lsection{Objetivos, enfoque, metodología y plan de trabajo}

\newpage \thispagestyle{empty} % Pagina vacia
